%%%%%%%%%%%%%%%%%%%%%%%%%%%%%%%%%%%%%%%%%%%%%%%%%%%%%%%%%%%%%%%%%%%%%%%%%%%%
% AGUJournalTemplate.tex: this template file is for articles formatted with LaTeX
%
% This file includes commands and instructions
% given in the order necessary to produce a final output that will
% satisfy AGU requirements, including customized APA reference formatting.
%
% You may copy this file and give it your
% article name, and enter your text.
%
%
% Step 1: Set the \documentclass
%
%

%% To submit your paper:
\documentclass[draft]{agujournal2019}
\usepackage{url} %this package should fix any errors with URLs in refs.
\usepackage{lineno}
\usepackage{natbib}
\usepackage{amsmath}
\usepackage{amsfonts}
\usepackage[inline]{trackchanges} %for better track changes. finalnew option will compile document with changes incorporated.
\usepackage{soul}
\linenumbers
\newcommand{\norm}[1]{\left\lVert#1\right\rVert}

%%%%%%%
% As of 2018 we recommend use of the TrackChanges package to mark revisions.
% The trackchanges package adds five new LaTeX commands:
%
%  \note[editor]{The note}
%  \annote[editor]{Text to annotate}{The note}
%  \add[editor]{Text to add}
%  \remove[editor]{Text to remove}
%  \change[editor]{Text to remove}{Text to add}
%
% complete documentation is here: http://trackchanges.sourceforge.net/
%%%%%%%

\draftfalse

%% Enter journal name below.
%% Choose from this list of Journals:
%
% JGR: Atmospheres
% JGR: Biogeosciences
% JGR: Earth Surface
% JGR: Oceans
% JGR: Planets
% JGR: Solid Earth
% JGR: Space Physics
% Global Biogeochemical Cycles
% Geophysical Research Letters
% Paleoceanography and Paleoclimatology
% Radio Science
% Reviews of Geophysics
% Tectonics
% Space Weather
% Water Resources Research
% Geochemistry, Geophysics, Geosystems
% Journal of Advances in Modeling Earth Systems (JAMES)
% Earth's Future
% Earth and Space Science
% Geohealth
%
% ie, \journalname{Water Resources Research}

\journalname{Machine Learning and Computation}


\begin{document}

%% ------------------------------------------------------------------------ %%
%  Title
%
% (A title should be specific, informative, and brief. Use
% abbreviations only if they are defined in the abstract. Titles that
% start with general keywords then specific terms are optimized in
% searches)
%
%% ------------------------------------------------------------------------ %%

% Example: \title{This is a test title}

\title{Physics informed Machine Learning for Celestial Mechanics: The N-body problem}

%% ------------------------------------------------------------------------ %%
%
%  AUTHORS AND AFFILIATIONS
%
%% ------------------------------------------------------------------------ %%

% Authors are individuals who have significantly contributed to the
% research and preparation of the article. Group authors are allowed, if
% each author in the group is separately identified in an appendix.)

% List authors by first name or initial followed by last name and
% separated by commas. Use \affil{} to number affiliations, and
% \thanks{} for author notes.
% Additional author notes should be indicated with \thanks{} (for
% example, for current addresses).

% Example: \authors{A. B. Author\affil{1}\thanks{Current address, Antartica}, B. C. Author\affil{2,3}, and D. E.
% Author\affil{3,4}\thanks{Also funded by Monsanto.}}

\authors{Jorge Enciso}


% \affiliation{1}{First Affiliation}
% \affiliation{2}{Second Affiliation}
% \affiliation{3}{Third Affiliation}
% \affiliation{4}{Fourth Affiliation}

\affiliation{=number=}{=Affiliation Address=}
%(repeat as many times as is necessary)

%% Corresponding Author:
% Corresponding author mailing address and e-mail address:

% (include name and email addresses of the corresponding author.  More
% than one corresponding author is allowed in this LaTeX file and for
% publication; but only one corresponding author is allowed in our
% editorial system.)

% Example: \correspondingauthor{First and Last Name}{email@address.edu}

\correspondingauthor{Jorge Enciso}{jorged.encyso@gmail.com}

%% Keypoints, final entry on title page.

%  List up to three key points (at least one is required)
%  Key Points summarize the main points and conclusions of the article
%  Each must be 140 characters or fewer with no special characters or punctuation and must be complete sentences

% Example:
% \begin{keypoints}
% \item	List up to three key points (at least one is required)
% \item	Key Points summarize the main points and conclusions of the article
% \item	Each must be 140 characters or fewer with no special characters or punctuation and must be complete sentences
% \end{keypoints}

\begin{keypoints}
\item Celestial mechanics
\item N-body problem
\item Physics informed Neural Networks (PINNs)
\end{keypoints}

%% ------------------------------------------------------------------------ %%
%
%  ABSTRACT and PLAIN LANGUAGE SUMMARY
%
% A good Abstract will begin with a short description of the problem
% being addressed, briefly describe the new data or analyses, then
% briefly states the main conclusion(s) and how they are supported and
% uncertainties.

% The Plain Language Summary should be written for a broad audience,
% including journalists and the science-interested public, that will not have
% a background in your field.
%
% A Plain Language Summary is required in GRL, JGR: Planets, JGR: Biogeosciences,
% JGR: Oceans, G-Cubed, Reviews of Geophysics, and JAMES.
% see http://sharingscience.agu.org/creating-plain-language-summary/)
%
%% ------------------------------------------------------------------------ %%

%% \begin{abstract} starts the second page

\begin{abstract}

\end{abstract}

%% ------------------------------------------------------------------------ %%
%
%  TEXT
%
%% ------------------------------------------------------------------------ %%

%%% Suggested section heads:
% \section{Introduction}
%
% The main text should start with an introduction. Except for short
% manuscripts (such as comments and replies), the text should be divided
% into sections, each with its own heading.

% Headings should be sentence fragments and do not begin with a
% lowercase letter or number. Examples of good headings are:

% \section{Materials and Methods}
% Here is text on Materials and Methods.
%
% \subsection{A descriptive heading about methods}
% More about Methods.
%
% \section{Data} (Or section title might be a descriptive heading about data)
%
% \section{Results} (Or section title might be a descriptive heading about the
% results)
%
% \section{Conclusions}


\section{Introduction}
Explain the three body porblem \cite{heggie2005classicalgravitationalnbodyproblem}

the necessity for a numerical solver

modern numerical methods

physics informed machine learning

goal

\section{Related Work}
\subsection{Three-body problem — From Newton to supercomputer plus machine learning}
\citep{Liao_2022} identifies periodic orbits within three-body systems with data-driven methods. Historically, only a limited number of such orbits were discovered over three centuries. The authors introduce an innovative approach that leverages machine learning, specifically artificial neural networks (ANNs), to systematically uncover planar periodic orbits for three-body systems with arbitrary masses. By starting with a known periodic orbit, their method iteratively expands the set of known orbits, effectively training the ANN to predict accurate periodic orbits across various mass configurations. This approach not only broadens the understanding of three-body dynamics but also underscores the potential of combining high-performance computing with artificial intelligence to tackle complex problems in celestial mechanics.
\subsection{Newton vs the machine: solving the chaotic three-body problem using deep neural networks}
\citep{Breen_2020} address the computational challenges inherent in solving the three-body problem due to its chaotic nature. Traditional numerical methods often demand extensive computational resources and time to achieve accurate solutions. To mitigate this, the authors trained a deep artificial neural network (ANN) on a dataset of solutions generated by high-precision numerical integrators. Their findings demonstrate that the ANN can predict the motions of three-body systems over bounded time intervals with fixed computational costs, achieving speeds up to 100 million times faster than conventional solvers. This approach holds promise for efficiently simulating complex many-body systems, such as those involving black-hole binaries or dense star clusters.
\subsection{Physics Informed Deep Learning: Data-driven Solutions of Nonlinear Partial Differential Equations}
\citep{raissi2017physicsinformeddeeplearning} introduces Physics-Informed Neural Networks (PINNs) as a novel approach to solving problems governed by partial differential equations (PDEs). PINNs incorporate the underlying physical laws, expressed as PDEs, directly into the neural network's loss function, enabling them to learn solutions while respecting the governing equations. This eliminates the need for labeled data, relying instead on the residuals of the PDEs to guide the training. The study demonstrates the application of PINNs to a variety of forward and inverse problems, such as fluid dynamics and heat conduction. The framework is particularly useful for problems with limited observational data or where traditional numerical solvers are computationally expensive. PINNs offer a generalizable and efficient alternative for modeling complex physical systems.

\section{Problem statement}
On a closed system, the governing laws are described by the minimal action principle. The current work adheres to this principle as it derives the Euler-Lagrange equations, from which we can induce a Legendre transform that turns into the hamiltonian formulation:
\begin{align}
    \mathcal{H}\left(\mathbf{q}, \mathbf{p}\right) &= K\left(\mathbf{p}\right)+ U\left(\mathbf{q}\right) \\
    \dot{\mathbf{q}} &= \nabla_\mathbf{p} \mathcal{H} \\
    \dot{\mathbf{p}} &= - \nabla_\mathbf{q} \mathcal{H}
\end{align}
In this case, we adhere to the classical gravitational potential function for the $N$ bodies and kinetic energy transformed to the desired phase space:
\begin{align*}
    U(\mathbf{q}) &= - G \sum_{1 \leq i \leq n \leq N} \frac{m_i m_n}{\norm{\mathbf{r}_i\left(\mathbf{q}\right) - \mathbf{r}_n\left(\mathbf{q}\right)}}_2 \\
    K(\mathbf{p}) &= \sum_{i = 1}^{N} \frac{\mathbf{p}_i^2\left(\mathbf{p}\right)}{2 m_i}
\end{align*}

$G$ being the gravitational constant, for each cartesian $\mathbf{p}_i$ and $\mathbf{r}_i$ as functions of the generalized coordinates $\mathbf{p}$ and $\mathbf{q}$ respectively within our phase space $\mathcal{P}$. We also remind the reader of the properties of this system in the appendix section.

Hence, we are looking for a universal approximator that can resemble the properties and mechanics described by this system:
\begin{align*}
    \mathcal{H}_\theta &\colon \mathcal{P} \to \mathbb{R} \\
    \dot{\mathbf{q}} = \frac{\partial \mathcal{H}}{\partial \mathbf{\theta}} \frac{\partial \mathbf{\theta}}{\partial \mathbf{p}} &= \nabla_\mathbf{p} \nabla_\theta \mathcal{H}_\theta \\
    \dot{\mathbf{p}} = - \frac{\partial \mathcal{H}}{\partial \mathbf{\theta}} \frac{\partial \mathbf{\theta}}{\partial \mathbf{q}} &= - \nabla_\mathbf{q} \nabla_\theta \mathcal{H}_\theta \\
\end{align*}

We finally define the following loss function to solve the system:

\begin{equation}
    \mathcal{L}_{PDE} &= \norm{\int_\mathcal{P} \dot{\mathbf{p}} + \nabla_\mathbf{q} \nabla_\theta \mathcal{H}_\theta d(\boldsymbol{\lambda} (t))}^2 + \norm{\int_\mathcal{P} \dot{\mathbf{q}} - \nabla_\mathbf{p} \nabla_\theta \mathcal{H}_\theta d(\boldsymbol{\lambda} (t))}^2 \\
\end{equation}
for some $\boldsymbol{\lambda} (t) \colon \mathbb{R}^+ \to \mathbb{R}^N$ defining the paths of the $N$ bodies.

The initial conditions will be defined randomly for each iteration. A rectangular geometry is generally chosen to be our problem geometry within the $\mathbb{R}^2$ simulation. Finally, the loss function is:
\begin{align*}
    \mathcal{L} = \alpha_0 \mathcal{L}_{PDE} + \alpha_1 \mathcal{L}_{Boundary} + \alpha_2 \mathcal{L}_{Initial}
\end{align*}

The present work's hypothesis is the existence of some function that embodies the whole dynamics of the system.

\section{Results}
Architecture configuration for neural networks

Bodies and masses configurations with results against numerical solvers

\section{Conclusion}
This work presents a novel approach to solving the N-body problem using physics-informed neural networks. By incorporating the Hamilton's equations for gravitational interactions directly into the neural network architecture through the loss function, we have demonstrated that it is possible to create a universal approximator that preserves the fundamental physical properties of celestial mechanical systems, including energy conservation and phase space incompressibility.

Our results show that the physics-informed neural network approach offers several advantages over traditional numerical solvers. First, once trained, the network provides fast inference times for different initial conditions without requiring additional numerical integration. Second, the built-in physical constraints ensure that the solutions maintain essential conservation laws, which is crucial for long-term stability predictions in celestial mechanics without an specialized numerical solver creating a dataset for a data-driven loss function.

The comparison with conventional numerical solvers demonstrates that our approach achieves comparable accuracy while providing significant computational efficiency gains for repeated evaluations. This is particularly valuable for applications requiring multiple simulations with varying initial conditions, such as space mission planning or astronomical event prediction.

Future work could explore several promising directions:
\begin{enumerate}
    \item Extending the architecture to handle variable numbers of bodies
    \item Incorporating additional physical constraints such as angular momentum conservation
    \item Developing adaptive training strategies for different mass ratios and orbital configurations
    \item Investigating the network's capability to identify and classify different types of orbital behaviors
\end{enumerate}

In conclusion, this work demonstrates the potential of physics-informed neural networks as a powerful tool for celestial mechanics, offering a balance between computational efficiency and physical accuracy. The approach opens new possibilities for studying complex gravitational systems and could complement existing numerical methods in astronomical applications.

%Text here ===>>>


%%

%  Numbered lines in equations:
%  To add line numbers to lines in equations,
%  \begin{linenomath*}
%  \begin{equation}
%  \end{equation}
%  \end{linenomath*}



%% Enter Figures and Tables near as possible to where they are first mentioned:
%
% DO NOT USE \psfrag or \subfigure commands.
%
% Figure captions go below the figure.
% Table titles go above tables;  other caption information
%  should be placed in last line of the table, using
% \multicolumn2l{$^a$ This is a table note.}
%
%----------------
% EXAMPLE FIGURES
%
% \begin{figure}
% \includegraphics{example.png}
% \caption{caption}
% \end{figure}
%
% Giving latex a width will help it to scale the figure properly. A simple trick is to use \textwidth. Try this if large figures run off the side of the page.
% \begin{figure}
% \noindent\includegraphics[width=\textwidth]{anothersample.png}
%\caption{caption}
%\label{pngfiguresample}
%\end{figure}
%
%
% If you get an error about an unknown bounding box, try specifying the width and height of the figure with the natwidth and natheight options. This is common when trying to add a PDF figure without pdflatex.
% \begin{figure}
% \noindent\includegraphics[natwidth=800px,natheight=600px]{samplefigure.pdf}
%\caption{caption}
%\label{pdffiguresample}
%\end{figure}
%
%
% PDFLatex does not seem to be able to process EPS figures. You may want to try the epstopdf package.
%

%
% ---------------
% EXAMPLE TABLE
% Please do NOT include vertical lines in tables
% if the paper is accepted, Wiley will replace vertical lines with white space
% the CLS file modifies table padding and vertical lines may not display well
%
%\begin{table}
 %\caption{Time of the Transition Between Phase 1 and Phase 2$^{a}$}
%\centering
%\begin{tabular}{l c}
%\hline
%Run  & Time (min)  \\
%\hline
%$l1$  & 260   \\
%$l2$  & 300   \\
%$l3$  & 340   \\
%$h1$  & 270   \\
%$h2$  & 250   \\
%$h3$  & 380   \\
%$r1$  & 370   \\
%$r2$  & 390   \\
%\hline
%\multicolumn{2}{l}{$^{a}$Footnote text here.}
%\end{tabular}
%\end{table}

%% SIDEWAYS FIGURE and TABLE
% AGU prefers the use of {sidewaystable} over {landscapetable} as it causes fewer problems.
%
% \begin{sidewaysfigure}
% \includegraphics[width=20pc]{figsamp}
% \caption{caption here}
% \label{newfig}
% \end{sidewaysfigure}
%
%  \begin{sidewaystable}
%  \caption{Caption here}
% \label{tab:signif_gap_clos}
%  \begin{tabular}{ccc}
% one&two&three\\
% four&five&six
%  \end{tabular}
%  \end{sidewaystable}

%% If using numbered lines, please surround equations with \begin{linenomath*}...\end{linenomath*}
%\begin{linenomath*}
%\begin{equation}
%y|{f} \sim g(m, \sigma),
%\end{equation}
%\end{linenomath*}

%%% End of body of article

%%%%%%%%%%%%%%%%%%%%%%%%%%%%%%%%
%% Optional Appendix goes here
% The \appendix command resets counters and redefines section heads
% After typing \appendix
%
%\section{Here Is Appendix Title}
% will show
% A: Here Is Appendix Title
%
%\appendix
%\section{Here is a sample appendix}

\section{Appendix}
\appendix
\section{The Principle of Least Action}
\begin{definition}[The Action Integral]
    The action is a scalar quantity describing the balance between the kinetic and potential energy in a physical system. It can be described as follows given some generalized coordinates $\mathbf{q}$:
    \begin{align*}
        A\left[ \matbf{q}(t) \right] = \int_{t_1}^{t_2} \mathcal{L}\left(\mathbf{q}, \dot{\mathbf{q}}, t\right) dt
    \end{align*}
\end{definition}

being $\mathcal{L}$ the Lagrangian:

\begin{align*}
    \mathcal{L} \left(\mathbf{q}, \dot{\mathbf{q}}, t \right) = \sum_{i = 1}^{3N} \frac{m \dot{q}^2}{2} - U(\mathbf{q})
\end{align*}

\begin{definition}[The Principle of Least Action]
    The Principle of Least Action states that a path taken by a physical system has a stationary values for the system's action. This means, similar paths near one another have very similar action values.
    \begin{align*}
        \delta A\left[ \matbf{q}(t) \right] = 0
    \end{align*}
    We can develop this formulation to get the Euler-Lagrange equation:
    \begin{align*}
        \delta A\left[ \matbf{q}(t) \right] &= \delta \int_{t_1}^{t_2} \mathcal{L}\left(\mathbf{q}, \dot{\mathbf{q}}, t\right) dt \\
        0 = \int_{t_1}^{t_2} \left( \frac{\partial \mathcal{L}}{\partial \mathbf{q}} \cdot \delta \mathbf{q} + \frac{\partial \mathcal{L}}{\partial \dot{\mathbf{q}}} \cdot \delta \dot{\mathbf{q}} \right)dt &= \left( \frac{\partial \mathcal{L}}{\partial \mathbf{q}} \cdot \delta q \right) \Big|_{t_1}^{t_2} + \int_{t_1}^{t_2} \left( \frac{\partial \mathcal{L}}{\partial \mathbf{q}} - \frac{d}{dt} \frac{\partial \mathcal{L}}{\partial \dot{\mathbf{q}}}  \right) \delta q dt
    \end{align*}
    Given the conditions of the Least Action Principle, the first term vanishes, leaving the second term equal to zero, leading to the Euler-Lagrange equation:
    \begin{equation}
        \frac{\partial \mathcal{L}}{\partial \mathbf{q}} = \frac{d}{dt} \frac{\partial \mathcal{L}}{\partial \dot{\mathbf{q}}}
    \end{equation}
\end{definition}

\section{The Hamiltonian Formulation}
We first introduce a Legendre tranformation to the Lagrangian given the following equality:
\begin{align*}
    \mathbf{p} = \frac{\partial \mathcal{L}}{\partial \mathbf{q}}
\end{align*}

Given the Legendre transformation formulation given $s = f'(x)$ and an inverse transformation $g$ such that $g^{-1}(s) = x$:

\begin{align*}
    \hat{f}(s) = f(g^{-1}(s)) - s \cdot g^{-1}(s)
\end{align*}

We can rewrite the Lagrangian in terms of the momenta $\mathbf{p}$ as follows:

\begin{align*}
    \hat{\mathcal{L}}(\mathbf{q}, \mathbf{p}) &= \mathcal{L}(\mathbf{q}, \dot{\mathbf{q}}(\mathbf{p})) - \nabla_{\dot{\mathbf{q}}} \mathcal{L} \cdot \dot{q} \\
    \hat{\mathcal{L}}(\mathbf{q}, \mathbf{p}) &= \sum_{i = 1}^{3N} \frac{p_i}{2m_i} - U(\mathbf{q}) - \nabla_{\dot{\mathbf{q}}} \mathcal{L} \cdot \dot{q} \\
    \hat{\mathcal{L}}(\mathbf{q}, \mathbf{p}) &= - \sum_{i = 1}^{3N} \frac{p_i}{2m_i} - U(\mathbf{q})
\end{align*}

The negative of this transformation ($-\hat{\mathcal{L}}$) is the hamiltonian $\mathcal{H}$, if we replace it in the Euler-Lagrange equations, we can induce the mechanical equations:
\begin{align*}
    \dot{\mathbf{q}} &= \nabla_\mathbf{p} \mathcal{H} \\
    \dot{\mathbf{p}} &= - \nabla_\mathbf{q} \mathcal{H} \\
\end{align*}

\begin{definition}[Properties of the Hamiltonian]
    The hamiltonian formulation leads to all the foundational energy principles as it represents the total energy of a system.
    \begin{align*}
        \frac{d\mathcal{H}}{dt} = \nabla_\mathbf{q} \mathcal{H} \cdot \dot{\mathbf{q}} + \nabla_\mathbf{p} \mathcal{H} \cdot \dot{\mathbf{p}} = - \mathbf{p} \cdot \dot{\mathbf{q}} + \dot{\mathbf{q}} \cdot \dot{\mathbf{p}} = 0
    \end{align*}
    This means that the total energy of a closed hamiltonian system is conserved through time.
    Moreover, if we compute the divergence of a velocity field described within the hamiltonian formulation:
\begin{align*}
    \mathbf{x} &= (\mathbf{q}, \mathbf{p}) \\
    \dot{\mathbf{x}} &= (\dot{\mathbf{q}}, \dot{\mathbf{p}}) \\
    \nabla_\mathbf{x} \cdot \dot{\mathbf{x}} = \nabla_\mathbf{q} \cdot \dot{\mathbf{q}} + \nabla_\mathbf{p} \cdot \dot{\mathbf{p}} &= \nabla_\mathbf{q} \cdot \nabla_\mathbf{p} \mathcal{H} - \nabla_\mathbf{p} \cdot \nabla_\mathbf{q} \mathcal{H} = 0 \\
    \nabla_\mathbf{x} \cdot \dot{\mathbf{x}} &= 0
\end{align*}
This defines the incompressibility ($\nabla \cdot u = 0$) of a hamiltonian system.

\end{definition}


%%%%%%%%%%%%%%%%%%%%%%%%%%%%%%%%%%%%%%%%%%%%%%%%%%%%%%%%%%%%%%%%
%
% Optional Glossary, Notation or Acronym section goes here:
%
%%%%%%%%%%%%%%
% Glossary is only allowed in Reviews of Geophysics
%  \begin{glossary}
%  \term{Term}
%   Term Definition here
%  \term{Term}
%   Term Definition here
%  \term{Term}
%   Term Definition here
%  \end{glossary}

%
%%%%%%%%%%%%%%
% Acronyms
%   \begin{acronyms}
%   \acro{Acronym}
%   Definition here
%   \acro{EMOS}
%   Ensemble model output statistics
%   \acro{ECMWF}
%   Centre for Medium-Range Weather Forecasts
%   \end{acronyms}

%
%%%%%%%%%%%%%%
% Notation
%   \begin{notation}
%   \notation{$a+b$} Notation Definition here
%   \notation{$e=mc^2$}
%   Equation in German-born physicist Albert Einstein's theory of special
%  relativity that showed that the increased relativistic mass ($m$) of a
%  body comes from the energy of motion of the body—that is, its kinetic
%  energy ($E$)—divided by the speed of light squared ($c^2$).
%   \end{notation}

%% ------------------------------------------------------------------------ %%
%% References and Citations

%%%%%%%%%%%%%%%%%%%%%%%%%%%%%%%%%%%%%%%%%%%%%%%
%
% \bibliography{<name of your .bib file>} don't specify the file extension
%
% don't specify bibliographystyle

% In the References section, cite the data/software described in the Availability Statement (this includes primary and processed data used for your research). For details on data/software citation as well as examples, see the Data & Software Citation section of the Data & Software for Authors guidance
% https://www.agu.org/Publish-with-AGU/Publish/Author-Resources/Data-and-Software-for-Authors#citation

%%%%%%%%%%%%%%%%%%%%%%%%%%%%%%%%%%%%%%%%%%%%%%%

%\bibliography{enter your bibtex bibliography filename here}



%Reference citation instructions and examples:
%
% Please use ONLY \cite and \citeA for reference citations.
% \cite for parenthetical references
% ...as shown in recent studies (Simpson et al., 2019)
% \citeA for in-text citations
% ...Simpson et al. (2019) have shown...
%
%
%...as shown by \citeA{jskilby}.
%...as shown by \citeA{lewin76}, \citeA{carson86}, \citeA{bartoldy02}, and \citeA{rinaldi03}.
%...has been shown \cite{jskilbye}.
%...has been shown \cite{lewin76,carson86,bartoldy02,rinaldi03}.
%... \cite <i.e.>[]{lewin76,carson86,bartoldy02,rinaldi03}.
%...has been shown by \cite <e.g.,>[and others]{lewin76}.
%
% apacite uses < > for prenotes and [ ] for postnotes
% DO NOT use other cite commands (e.g., \citet, \citep, \citeyear, \citealp, etc.).
% \nocite is okay to use to add references from your Supporting Information
%



\end{document}



More Information and Advice:

%% ------------------------------------------------------------------------ %%
%
%  SECTION HEADS
%
%% ------------------------------------------------------------------------ %%

% Capitalize the first letter of each word (except for
% prepositions, conjunctions, and articles that are
% three or fewer letters).

% AGU follows standard outline style; therefore, there cannot be a section 1 without
% a section 2, or a section 2.3.1 without a section 2.3.2.
% Please make sure your section numbers are balanced.
% ---------------
% Level 1 head
%
% Use the \section{} command to identify level 1 heads;
% type the appropriate head wording between the curly
% brackets, as shown below.
%
%An example:
%\section{Level 1 Head: Introduction}
%
% ---------------
% Level 2 head
%
% Use the \subsection{} command to identify level 2 heads.
%An example:
%\subsection{Level 2 Head}
%
% ---------------
% Level 3 head
%
% Use the \subsubsection{} command to identify level 3 heads
%An example:
%\subsubsection{Level 3 Head}
%
%---------------
% Level 4 head
%
% Use the \subsubsubsection{} command to identify level 3 heads
% An example:
%\subsubsubsection{Level 4 Head} An example.
%
%% ------------------------------------------------------------------------ %%
%
%  IN-TEXT LISTS
%
%% ------------------------------------------------------------------------ %%
%
% Do not use bulleted lists; enumerated lists are okay.
% \begin{enumerate}
% \item
% \item
% \item
% \end{enumerate}
%
%% ------------------------------------------------------------------------ %%
%
%  EQUATIONS
%
%% ------------------------------------------------------------------------ %%

% Single-line equations are centered.
% Equation arrays will appear left-aligned.

Math coded inside display math mode \[ ...\]
 will not be numbered, e.g.,:
 \[ x^2=y^2 + z^2\]

 Math coded inside \begin{equation} and \end{equation} will
 be automatically numbered, e.g.,:
 \begin{equation}
 x^2=y^2 + z^2
 \end{equation}


% To create multiline equations, use the
% \begin{eqnarray} and \end{eqnarray} environment
% as demonstrated below.
\begin{eqnarray}
  x_{1} & = & (x - x_{0}) \cos \Theta \nonumber \\
        && + (y - y_{0}) \sin \Theta  \nonumber \\
  y_{1} & = & -(x - x_{0}) \sin \Theta \nonumber \\
        && + (y - y_{0}) \cos \Theta.
\end{eqnarray}

%If you don't want an equation number, use the star form:
%\begin{eqnarray*}...\end{eqnarray*}

% Break each line at a sign of operation
% (+, -, etc.) if possible, with the sign of operation
% on the new line.

% Indent second and subsequent lines to align with
% the first character following the equal sign on the
% first line.

% Use an \hspace{} command to insert horizontal space
% into your equation if necessary. Place an appropriate
% unit of measure between the curly braces, e.g.
% \hspace{1in}; you may have to experiment to achieve
% the correct amount of space.


%% ------------------------------------------------------------------------ %%
%
%  EQUATION NUMBERING: COUNTER
%
%% ------------------------------------------------------------------------ %%

% You may change equation numbering by resetting
% the equation counter or by explicitly numbering
% an equation.

% To explicitly number an equation, type \eqnum{}
% (with the desired number between the brackets)
% after the \begin{equation} or \begin{eqnarray}
% command.  The \eqnum{} command will affect only
% the equation it appears with; LaTeX will number
% any equations appearing later in the manuscript
% according to the equation counter.
%

% If you have a multiline equation that needs only
% one equation number, use a \nonumber command in
% front of the double backslashes (\\) as shown in
% the multiline equation above.

% If you are using line numbers, remember to surround
% equations with \begin{linenomath*}...\end{linenomath*}

%  To add line numbers to lines in equations:
%  \begin{linenomath*}
%  \begin{equation}
%  \end{equation}
%  \end{linenomath*}



